% In this file you should put the actual content of the blueprint.
% It will be used both by the web and the print version.
% It should *not* include the \begin{document}
%
% If you want to split the blueprint content into several files then
% the current file can be a simple sequence of \input. Otherwise It
% can start with a \section or \chapter for instance.

\begin{lemma}\label{minpoly_deg_le_two}\lean{minpoly_deg_le_two}\leanok
    最小多項式 \(m_M(x)\) の次数は \(2\) 以下.
\end{lemma}

\begin{lemma}\label{minpoly_dvd_X_pow_sub_one}\lean{minpoly_dvd_X_pow_sub_one}\leanok
    最小多項式 \(m_M(x)\) は, \(x^n - 1\) を割り切る.
\end{lemma}

\begin{lemma}\label{exist_normalizefactor}\lean{exist_normalizefactor}\leanok
    体上の単元でない任意の多項式にはモニックな既約因子が存在する.
\end{lemma}

\begin{lemma}\label{normalizedfactor_eq_cyclotomic}\lean{normalizedfactor_eq_cyclotomic}\leanok\uses{minpoly_dvd_X_pow_sub_one}
    最小多項式 \(m_M(x)\) のモニックな既約因子は, ある \(n\) の正の約数 \(i\) に対する
    第 \(i\) 円分多項式 \(\Phi_i(x)\) と一致する.
\end{lemma}

\begin{lemma}\label{cyclotomic_deg_eq_totient}\lean{cyclotomic_deg_eq_totient}\leanok
    第 \(n\) 円分多項式 \(\Phi_n(x)\) の次数は, オイラーのトーシェント関数 \(\phi(n)\) に等しい.
\end{lemma}

\begin{lemma}\label{n_exist}\lean{n_exist}\leanok\uses{totient_factorization, exist_factorization}
    \(\phi(n) = 2\) ならば, \(n\) は \(2\) と \(3\) 以外の素因数をもたない.
\end{lemma}

\begin{lemma}\label{totient_eq_two_iff}\lean{totient_eq_two_iff}\leanok\uses{n_exist}
    \(\phi(n) = 2\) ~\Leftrightarrow~ \(n = 3, 4, 6\).
\end{lemma}

\begin{lemma}\label{totient_le_two_iff}\lean{totient_le_two_iff}\leanok\uses{totient_eq_two_iff}
    \(\phi(n) \le 2\) ~\Leftrightarrow~ \(n = 1, 2, 3, 4, 6\).
\end{lemma}

\begin{lemma}\label{cyclotomic_four}\lean{cyclotomic_four}\leanok
    \(\Phi_4(x) = x^2 + 1\).
\end{lemma}

\begin{lemma}\label{cyclotomic_six}\lean{cyclotomic_six}\leanok
    \(\Phi_6(x) = x^2 - x + 1\).
\end{lemma}

\begin{lemma}\label{normalizedfactor_class}\lean{normalizedfactor_class}\leanok\uses{normalizedfactor_eq_cyclotomic, minpoly_deg_le_two, cyclotomic_deg_eq_totient, totient_le_two_iff}
    最小多項式のモニックな既約因子は
    \[\Phi_1(x), \Phi_2(x), \Phi_3(x), \Phi_4(x), \Phi_6(x) \quad \text{に限る.}\]
\end{lemma}

\begin{lemma}\label{minpoly_squarefree}\lean{minpoly_squarefree}\leanok\uses{minpoly_dvd_X_pow_sub_one}
    最小多項式 \(m_M(x)\) は重解を持たない.
\end{lemma}

\begin{lemma}\label{minpoly_class}\lean{minpoly_class}\leanok\uses{exist_normalizefactor, normalizedfactor_class, minpoly_deg_le_two, minpoly_squarefree, cyclotomic_deg_eq_totient, cyclotomic_four, cyclotomic_six}
    \(M\) の最小多項式 \(m_M(x)\) は
    \[\Phi_1(x), \Phi_2(x), \Phi_3(x), \Phi_4(x), \Phi_6(x), \Phi_1(x)\Phi_2(x) \quad \text{に限る.}\]
\end{lemma}

\begin{lemma}\label{minpoly_cyc_order}\lean{minpoly_cyc_order}\leanok\uses{cyclotomic_four, cyclotomic_six}
    \(m_M(x) = \Phi_n(x)\) ならば, \(M\) の位数は \(n\) に等しい.
\end{lemma}

\begin{theorem}\label{finorder_class}\lean{finorder_class}\leanok\uses{minpoly_class, minpoly_cyc_order}
    \(GL(2, \mathbb{Q})\) の有限位数の元の位数は \(1, 2, 3, 4, 6\) に限る.
\end{theorem}

\begin{definition}\label{toGL}\lean{toGL}\leanok
    有理数体上の \(2\) 次正方行列 \(M\) を \(GL(2, \mathbb{Q})\) の元とみなすための同型写像.
\end{definition}

\begin{theorem}\label{finite_order_matrix}\lean{finite_order_matrix}\leanok\uses{toGL}
    \(n = 1, 2, 3, 4, 6\) それぞれに対して, 位数が \(n\) となるような
    \(GL(2, \mathbb{Q})\) の元 \(M\) が存在する.
\end{theorem}